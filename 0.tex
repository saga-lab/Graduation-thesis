%#!platex --src-specials main.tex

\Title{日本語タイトル} 
\Author{熊本太郎}
\Date{2}{2}{19}

%学部の人は次のコメント行の%を外してください.
%\senior  %この\seniorは\Synopsisの前であればどこに書いてもかまいません.

\Synopsis

\begin{Abstract}
%アブスト

\end{Abstract}

%
% これより下は学部(卒業論文を書く人)には関係ありません. 
% 
\title{English Titile}
\author{Taro Kumamoto}
\endate{February}{2}{2021}

\synopsis

\begin{abstract}
English
\end{abstract}










% 修士論文の論文概要

% 修士論文については和文と英文の論文概要を次の要領で作成し,【5】のb.2として下
% さい。英文で本文を記述した場合も,論文概要は和文,英文の両方で作成することが望ま
% しい(詳細は指導教員の指示に従って下さい)。

% 1.論文概要(和文)の形式

%     a.修士論文と同じ体裁で作った表紙(図2)を付けます。ただし,題目と氏名の間
%     に,「論文概要」と書き添えて下さい。

%     b.研究の目的,論文全体のあらまし,各章の内容(簡単に),結論(やや詳し
%     く),得られた成果の意義を,この順序で3~5ページ程度にまとめて下さい。

% 2.論文概要(英文)の形式

%    a.修士論文と同じ体裁で作った表紙(図2)を英文で記載して付けます。ただし,題
%    目と氏名の間に, Synopsis と書き添えて下さい

%    b.研究の目的,論文全体のあらまし,結論を,100~300 語程度にまとめて下さい。


% 3.和文,英文ともに目次は付けないで下さい。また,原則として,図,表,式を用いな
% いで下さい。
